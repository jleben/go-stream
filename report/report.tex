\documentclass {article}
\usepackage{graphicx}
\usepackage{subfigure}
\usepackage{listings}
\usepackage{courier}
\usepackage[top=2.5cm, bottom=2.5cm, left=2.5cm, right=2.5cm]{geometry}

\begin {document}
\title{UVic CSC 564 Concurrency\\Assignment 2 Report\\Go Language Evaluation}
\author{Jakob Leben}
\date{March 2014}
\maketitle

\lstset{basicstyle=\small\fontfamily{pcr}\selectfont}

\section{Introduction}

The goal of this assignment was to get familiar and experiment with the Go
programming language, and evaluate its advantages and drawbacks. I thought the
best way to achieve this was to try and develop code that can serve a real
purpose and actually be put to good use beyond the scope of this assignment.

This report includes the following sections:

\begin{description}

\item[\ref{sec:project}: Project Description] A programming project implemented
in Go is described. Reasons are given why this particular project is relevant
for evaluation of Go. Main project goals are defined, challenges faced by any
implementation are described, and an overview of the solutions in Go is
provided.

\item[\ref{sec:language}: Go Language Evaluation] The language is evaluated
based on the details of implementation of the above-defined project. Comparison
is drawn to other languages.

\item[\ref{sec:runtime}: Go Runtime Evaluation] The code is put into use and
performance is measured in various ways. Reflection is provided on how the
performance characteristics are a necessary consequence of the essential
language design, what could potentially be done differently and what tradeoffs
between performance qualities and qualities of the language are at play.

\end{description}

\section{Project Description}
\label{sec:project}

\subsection{Overview of Go and Motivation}

In support of the choice of particular programming project, let me first provide
a quick overview of the most prominent characteristics of the Go language. The
most distinguishing features of Go are arguably "goroutines" and "channels".

Goroutines are an embodiment of the concept of co-routines found in other
languages: each goroutine is a chunk of sequential code, but different
goroutines can run concurrently. They are first-class concepts in the Go
language and are thus distinguished from the operating system concept of
threads. Goroutines may indeed run in parallel on different threads, or they may
run on a single thread in an interleaved fashion where only one at a time ever
progresses but they take turns in doing so. The assumption is that the way
goroutines are actually executed is irrelevant for the task of program
implementation, and the Go runtime together with the operating system should
ideally ensure the best possible execution of any given set of goroutines.

Channels are facilities for orderly and deterministic communication of otherwise
independently running goroutines. In the essence, they correspond to channels as
described by Hoare in Communicating Sequential Processes (CSP) \ref{bib:csp}. A
channel allows a goroutine to send information to another goroutine. It also
allows for goroutine synchronization: the sending goroutine will wait until the
sent information is received, and the receiving one will wait until the expected
information is sent. Moreover, Go's channels feature several extensions of the
CSP channels. They can optionally be buffered with a limited buffer size, so
sending and receiving will only block if the channel is full or empty,
respectively. Unlike channels in CSP which are defined implicitly by addressing
send and receive operations to named processes, channels in Go are themselves
named objects that can be passed as information just like any other object.
Hence it is possible to have channels of channels, which allows for complex
communication patterns.

However, channels do allow for some degree of indeterminism. Specifically, there
is a \lstinline|select| statement which groups several send and receive
operations, waits until any of them can complete, and proceeds to execute one
such operation. If more than one operation could complete, one is randomly
chosen. Moreover, multiple goroutines can write to and read from the same
channel concurrently. In that case, individual send and receive operations are
paired, but it is undefined which send will be paired with which receive.

\subsection{Music and Streams}

One area of application to which communicating concurrent processes fit quite
naturally is music. We usually think of music as composed of individual
cooperating voices: each voice belongs to a distinct sound source with specific
characteristics, a sound source is producing sounds sequentially in time, while
a collection of such sources cooperate in a well organized fashion. It is this
organization that results in a pleasing perceptual effect which we call
\emph{music}. Hence, a definition of a musical piece is a definition of how
sounds belonging to a voice follow one another in time, as well as how sounds of
different voices are combined in time. Musical meaning is essentially dual:
sequential and concurrent.

To enable algorithmic composition of music, we can think of an individual
musical voice as a continous stream of information produced by repeated
evaluation of a composition of primitive operators, some of which are simple
data sources and others are filters that map input data to output data. For this
reason most computer software for algorithmic music composition follows the
coarse-grained dataflow programming paradigm and shares many aspects with the
stream processing paradigm as defined in \ref{bib:streams}. Different voices are
the result of a concurrent combination of individual voice streams with
potential interdependencies: an operator contributing to the final output of one
stream may require as input information produced by an operator contributing to
the output of another stream. Moreover, the musical output of individual voices
may feature multiple aspects controlled by the program, and control of
individual aspects itself depends on a collection of interdependent concurrent
information streams.
The high-level description of a program for music generation may thus be viewed
as a directed graph where nodes correspond to operators and edges correspond to
streams. Finally, outputs of different groups of nodes constitute information
that controls individual voices of music.

If we turn out attention to the sequential aspect of the musical meaning, we may
realize that the current output of individual stream operators will often depend
not only on their current input, but also their previous input and output. There
is thus a necessity to carry over a state from one to another execution of
operatators. In some software, this is facilitated with a data structure
corresponding to an operator and sharing its lifetime, which the operator can
use to store and retrive data across executions (which naturally maps to the
identity of an operator with an object in object-oriented languages). However,
programming becomes much simpler if each operator is represented with a
co-routine: each execution of an operator corresponds to executing only a part
of its code that produces a single unit of output, after which the operator is
interrupted and other operators downstream get executed using the output
produced upstream. All the required operator state can thus be represented by
the data
variables local to the operator code. This is the approach taken by the
SuperCollider framework \ref{bib:sc} in its system of "patterns and streams".

It then becomes of crucial importance the way that the execution of operators is
interleaved and how data is passed between these executions. In SuperCollider,
co-routine execution and communication is structured in a hierarchical way. A
parent routine explicitly requests execution of a child routine and at the same
time can pass information to it. The execution of the parent is blocked until
the child explicitly yields execution, at which time it can also pass
information back to the parent. The execution of the parent then continues with
the availability of the information provided by the child. In Go language,
communication and execution control are more decoupled. Assuming a restriction
of all communcation to passing data over channels, despite the send and receive
operations being coordinated, they do not essentially dictate an immediate
transfer of control to the other end of communication.

The challenge approached by this project was to use the Go language to replicate
the SuperCollider's high-level concept of stream operators as co-routines.
Operators are thus represented by goroutines, and all communication between them
occurs over channels.

\subsection{Overview of Implementation}

The goal of the project was to implement a framework for algorithmic music
composition where concurrent streams of events describing musical voices are
generated by a composition of primitive stream operators (sources and filters).
The implementation was guided by two aspects of evaluation of the Go language
and runtime:

\begin{itemize}
\item How elegant a programming interface for music composition can be devised?
\item How efficient would the framework be in real-time execution for serious
music performance?
\end{itemize}

To affirm the solution's usefulness, the final goal was to produce the actual
sound output in real time. For this reason, a basic interface with the
SuperCollider sound synthesis functionality was added. Fortunately, the
SuperCollider framework is well divided into the algorithmic music composition
part (the SuperCollider language) which is substituted by this project, and the
sound synthesis part (the audio server) which is reused by this project.
Communication with the audio server for sound synthesis control is done using
the Open Sound Control (OSC) format \ref{bib:osc} over a UDP/IP connection.
Hence, my solution includes the translation of abstract musical events into OSC
messages and sending them to the SuperCollider synthesis server. An external Go
package for OSC message generation was used for this purpose.

The implementation is split into several Go packages, some of which are more
generally useful utilities:

\begin{description}
\item['stream' package:] Stream operator definition, composition and execution.
\item['schedule' package:] Scheduling system that maps events in logical musical
time into real time.
\item['priority\_queue' package:] Priority queue data structure used by several
components.
\item['supercollider' package:] Translation of abstract streams into streams of
events targetted for SuperCollider. Translation of such events into OSC format
and sending them to the SuperCollider sound synthesis server.
\end{description}


\section{Go Language Evaluation}
\label{sec:language}

\subsection {Interfaces, Generics and Polymorphism}

The stream processing project is a kind of project with a need for a simple and
minimalistic common programming interface to define common protocol of
cooperation among a number of elements (stream operators), where the concrete
and specific operation of different elements should be completely abstracted
way. There are two fundamental approaches to this task: polymorphism and generic
programming.

Go lacks support for generic programming but instead offers a lot of flexibility
in polymorphism. This is counterbalanced by the necessary loss of code clarity.
The flexibility comes in several forms. Firstly, there is support for
object-oriented programming, but methods for a type do not make part of the
type's definition - they are defined separately. This means that methods can be
added to any type by its user, without modifying the type definition as provided
by the author. Secondly, there is the concept of interfaces, which are sets of
all types for which particular methods are implemented. However, the fact that a
type implements an interface is also not explicitly specified at the type's
definition, but implicitly by having required methods implemented. Furthermore,
interfaces can themselves be types of variables, method arguments and method
return types; this also holds for the empty interface, which is satisfied by any
type! In consequence Go's polymorphism is just as expressive as for example
generic (template) programming in C++ or Java. On the other hand, it is rather
hard to deduce which interfaces a particular type implements based only on
looking at the code that mentions the type. The code author's intention with
respect to interface implementation is unclear.

In practice, I found Go's interfaces a very useful means of structuring code.
The complete interface for definition, cooperation and usage of stream
processing framework is expressed concisely in the following code:

\begin{lstlisting}
type Item interface {}

type Status int

const (
  Ok Status = iota;
  Closed
  Interrupted
)

type Writer interface {
  Push (Item) Status
  Close ()
}

type Reader interface {
  Pull () (Item, Status)
  Close ()
}

type Operator interface {
  Stream() Reader
}
\end{lstlisting}

I will only provide a basic explanation of the interface here and details will
be explained later. The Item is the smallest discrete unit of information
transferred in any stream. It is an empty interface, so any other type can be
used in its place. There is a set of constants defining the result status of
stream production and consumption operations. The Writer represents the
producing end of a stream. Its Push method sends the Item argument downstream.
The Reader represents the consuming end of the stream. Its Pull method removes
and returns an Item from the stream. The Close methods notify the other end that
no writing or reading will be performed anymore by this end, and this is
reflected in the Status returned by the Push or Pull methods on the other end.
The Operator represents a concrete definition of stream processing. It's duty is
to provide a method Stream which instantiates a concrete stream of values and
returns the Reader interface to it. The \emph{output} of upstream operators is
thus accesed
by downstream operators via the Operator interface. However, it is left to
concrete implementations of the Operator to define whether they require or
accept any upstream Operators and how they are to be provided.


\subsection{Channels}

One of the main project goals was to allow the framework user to define
operators as goroutines. The Stream method of the Operator should thus run a
goroutine which should then be able to provide stream data Item by Item via the
Reader interface. The obvious way to exchange data between goroutines while they
are running are Go's channels.

Since channels are primitive types in Go, there is also a special, elegant and
compact syntax for sending and receiving data over channels (the \lstinline|<-|
operator). Hence, in an initial attempt, I intended to expose channels through
the interface all the way up to the user of the framework. There was no Writer
and Reader interfaces, but the Stream methods of the Operator interface returned
a channel type, used for the upstream operator to send data downstream. However,
it quickly turned out that more complex communication was required, involving at
least two channels per stream. Hence, the communcation was wrapped by the Reader
and Writer interfaces.

The need for more complex communication arises from the fact that an operator's
goroutine will only end when all of its data has been produced and sent
downstream, but the downstream operator may decide to stop consuming upstream
data before it is all consumed. If there was no way to communicate this decision
upstream, the upstream goroutine would never end and its resources never
released; Go does not provide any external means to stop a goroutine.

Each stream is thus represented by two channels: one for downstream data
transmision (operator output to input) and another channel to signal end of
stream consumption upstream. The Writer's Push operation thus uses the
\lstinline|select| statement to concurrently attempt sending data on one channel
as well as receiving the end-of-consumption signal on the other channel,
whichever one succeeds. The Reader's Close() operation sends the
end-of-consumption signal upstream. In the other direction, the Reader's Pull()
method simply receives from the data channel, which will be interrupted when the
channel is closed by the upstream Writer's Close() method.

A stream was thus implemented by the Stream data structure in the following
code. Note that the same Stream type was aliased as StreamWriter and
StreamReader, which were used as receivers in implementation of methods for the
Writer and Reader interfaces, respectively:

\begin{lstlisting}
type Stream struct {
  // Transfer stream output downstream:
  data chan Item
  // Request from downstream for this stream to end:
  finish chan struct {}
}

type StreamWriter Stream
type StreamReader Stream

func (s *StreamWriter) Push (data Item) Status { ... }
func (s *StreamWriter) Close () { ... }
func (s *StreamReader) Pull () (Item, Status) { ... }
func (s *StreamReader) Close () { ... }
\end{lstlisting}

In conclusion, to provide a simple interface that ensures correct operation, the
ability to use the channel send/receive operator (\lstinline|<-|) at the user
level was lost. Nevertheless, Go allowed for succesful abstraction of the more
complex communication in a simple-enough interface composed of an intuitive set
of operations, which is reflected in expressive method names.


\subsection{Subclassing, Instantiation, Function Literals}

Go does not have any formal notion of constructors as special methods. Only
conventionally, a type \lstinline|T| would have an associated method
\lstinline|NewT| which would return a pointer to an allocated and initialized
instance of \lstinline|T|. This is much like object-oriented programming in
languages with no specific support for this paradigm (for example C). I can
imagine that this becomes cumbersome when combined with the paradigm of
modelling subclassing by including an instance of the superclass as a member of
the data structure of the subclass: the quasi-constructor of the superclass must
be explicitly called, but it is not a compile-time error if it is not called.

In the case of this project, this did not cause much trouble. In fact, because
all state specific to different operators could be part of their goroutine stack
greatly reduced the number of data structures involved and hence the need for
their allocation and initialization. Furthermore, I realized that only two
generic data structures can be used to represent any operator: one for source
operators that take no input, and one for filter operators that take inputs.
This is facilitated by the ability for functions to act as objects, and the
support for variadic functions that take any number of arguments. Thus, the two
data structures contain function objects of appropriates types which define
specific operators. The SourceOp struct represents a source operator and
contains a function object of type SourceFunc that only takes one Writer
parameter for an output stream, and the FilterOp struct contains a function
object of type FilterFunc that takes one Writer for an output and any number of
Readers for input
streams, and a slice of Operator objects representing upstream operators.

\begin{lstlisting}
type SourceFunc func ( Writer )

type SourceOp struct {
  work SourceFunc
}

type FilterFunc func ( Writer, ...Reader )

type FilterOp struct {
  work FilterFunc
  sources [] Operator
}
\end{lstlisting}

There is one constructor method for each of the structs. There is a number of
concrete operators implemented simply as wrapper functions that internally
create a function literal (in some languages known as "lambda function") to
define their specific processing and pass it to the Source or Filter
constructor:

\begin{lstlisting}
func Source ( work SourceFunc ) Operator {
  return & SourceOp { work }
}

func Filter ( work FilterFunc, sources... Operator ) Operator {
  return & FilterOp { work, sources }
}
\end{lstlisting}


\subsection{Goroutines}

There is a good reason for the distinction between the operator as the stream
processing definition (a kind of template), and actual streams of values as
instances of this definition: for example, one very useful meta-operator would
provide repetition of a finite stream production process of another operator. It
is thus necessary for the user to be able to provide a single time the generator
of the stream to be repeated, and for the repeater to instantiate the stream
multiple times. This is why the operator constructors only create data
structures that represent a composition of operators while the Stream method of
the Operator interface is intended to instantiate the stream production process
of an operator. Each stream instantiation of an operator in turn instantiates
streams of upstream operators.

The stream instantiation is represented by the instantiation of a goroutine
which operates on Reader interfaces to consume input, processes the input and
operates on a Writer interface to produce output. The implementations of the
Stream method for SourceOp and FilterOp do little more than instantiating a
goroutine which in turn calls their concrete processing function. In addition,
FilterOp obtains input streams by calling Stream() on upstream operators, and
both SourceOp and FilterOp create an instance of the Stream struct to return as
their output stream. These streams are passed as Writer and Reader interfaces to
the concrete processing functions. Moreover, both FilterOp and SourceOp
goroutines call Close on all input and output streams after the processing
functions end.

\subsection{Conclusions}

Using a well-defined interface for cooperation between stream producers and
consumers, goroutine management and communication appeared to be a no-brainer
and emerged very naturally and intuitively from the interface. I would conclude
that Go's facilities for concurrent programming foster good code design and
allow intuition about the operation of concurrent programs that helps writing
error-free code.

\section{Go Runtime Evaluation}
\label{sec:runtime}

\end {document}
