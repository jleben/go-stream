\documentclass {article}
\usepackage{graphicx}
\usepackage{subfigure}
\usepackage{listings}
\usepackage{courier}
\usepackage[top=2.5cm, bottom=2.5cm, left=2.5cm, right=2.5cm]{geometry}

\begin {document}
\title{UVic CSC 564 Concurrency\\Assignment 2 Report\\Go Language Evaluation}
\author{Jakob Leben}
\date{March 2014}
\maketitle

\lstset{basicstyle=\small\fontfamily{pcr}\selectfont}

\section{Introduction}

The goal of this assignment was to get familiar and experiment with the Go programming language, and evaluate its advantages and drawbacks. I thought the best way to achieve this was to try and develop code that can serve a real purpose and actually be put to good use beyond the scope of this assignment.

This report includes the following sections:

\begin{description}

\item[\ref{sec:project}: Project Description] A programming project implemented in Go is described. Reasons are given why this particular project is relevant for evaluation of Go. Main project requirements are defined, challenges faced by any implementation are described, and solutions in Go language are explained.

\item[\ref{sec:language}: Go Language Evaluation] The language is evaluated based on the experience gained through implementation of the above described project. Comparison is drawn to other languages.

\item[\ref{sec:runtime}: Go Runtime Evaluation] The code is put into use and performance is measured in various ways. Reflection is provided on how the performance characteristics are a necessary consequence of the essential language design, what could potentially be done differently and what tradeoffs between performance qualities and qualities of the language are at play.

\end{description}

\section{Project Description}
\label{sec:project}

\subsection{Overview of Go and Motivation}

In support of the choice of particular programming project, let me first provide a quick overview of the most prominent characteristics of the Go language. The most distinguishing features of Go are arguably "goroutines" and "channels".

Goroutines are an embodiment of the concept of co-routines found in other languages: each goroutine is a chunk of sequential code, but different goroutines can run concurrently. They are first-class concepts in the Go language and are thus distinguished from the operating system concept of threads. Goroutines may indeed run in parallel on different threads, or they may run on a single thread in an interleaved fashion where only one at a time ever progresses but they take turns in doing so. The assumption is that the way goroutines are actually executed is irrelevant for the task of program implementation, and the Go runtime together with the operating system should ideally ensure the best possible execution of any given set of goroutines.

Channels are facilities for orderly and deterministic communication of otherwise independently running goroutines. In the essence, they correspond to channels as described by Hoare in Communicating Sequential Processes (CSP) \ref{bib:csp}. A channel allows a goroutine to send information to another goroutine. It also allows for goroutine synchronization: the sending goroutine will wait until the sent information is received, and the receiving one will wait until the expected information is sent. Moreover, Go's channels feature several extensions of the CSP channels. They can optionally be buffered with a limited buffer size, so sending and receiving will only block if the channel is full or empty, respectively. Unlike channels in CSP which are defined implicitly by addressing send and receive operations to named processes, channels in Go are themselves named objects that can be passed as information just like any other object. Hence it is possible to have channels of channels, which allows for complex communication patterns.

However, channels do allow for some degree of indeterminism. Specifically, there is a \lstinline|select| statement which groups several send and receive operations, waits until any of them can complete, and proceeds to execute one such operation. If more than one operation could complete, one is randomly chosen. Moreover, multiple goroutines can write to and read from the same channel concurrently. In that case, individual send and receive operations are paired, but it is undefined which send will be paired with which receive.

\subsection{Music and Streams}

One area of application to which communicating concurrent processes fit quite naturally is music. We usually think of music as composed of individual cooperating voices: each voice belongs to a distinct sound source with specific characteristics, a sound source is producing sounds sequentially in time, while a collection of such sources cooperate in a well organized fashion. It is this organization that results in a pleasing perceptual effect which we call \emph{music}. Hence, a definition of a musical piece is a definition of how sounds belonging to a voice follow one another in time, as well as how sounds of different voices are combined in time. Musical meaning is essentially dual: sequential and concurrent.

To enable algorithmic composition of music, we can think of an individual musical voice as a continous stream of information produced by repeated evaluation of a composition of primitive operators, some of which are simple data sources and others are filters that map input data to output data. For this reason most computer software for algorithmic music composition follows the coarse-grained dataflow programming paradigm and shares many aspects with the stream processing paradigm as defined in \ref{bib:streams}. Different voices are the result of a concurrent combination of individual voice streams with potential interdependencies: an operator contributing to the final output of one stream may require as input information produced by an operator contributing to the output of another stream. Moreover, the musical output of individual voices may feature multiple aspects controlled by the program, and control of individual aspects itself depends on a collection of interdependent concurrent information streams. The high-level description of a program for music generation may thus be viewed as a directed graph where nodes correspond to operators and edges correspond to streams. Finally, outputs of different groups of nodes constitute information that controls individual voices of music.

If we turn out attention to the sequential aspect of the musical meaning, we may realize that the current output of individual stream operators will often depend not only on their current input, but also their previous input and output. There is thus a necessity to carry over a state from one to another execution of operatators. In some software, this is facilitated with a data structure corresponding to an operator and sharing its lifetime, which the operator can use to store and retrive data across executions (which naturally maps to the identity of an operator with an object in object-oriented languages). However, programming becomes much simpler if each operator is represented with a co-routine: each execution of an operator corresponds to executing only a part of its code that produces a single unit of output, after which the operator is interrupted and other operators downstream get executed using the output produced upstream. All the required operator state can thus be represented by the data variables local to the operator code. This is the approach taken by the SuperCollider framework \ref{bib:sc} in its system of "patterns and streams".

It then becomes of crucial importance the way that the execution of operators is interleaved and how data is passed between these executions. In SuperCollider, co-routine execution and communication is structured in a hierarchical way. A parent routine explicitly requests execution of a child routine and at the same time can pass information to it. The execution of the parent is blocked until the child explicitly yields execution, at which time it can also pass information back to the parent. The execution of the parent then continues with the availability of the information provided by the child. In Go language, communication and execution control are more decoupled. Assuming a restriction of all communcation to passing data over channels, despite the send and receive operations being coordinated, they do not essentially dictate an immediate transfer of control to the other end of communication.

The challenge approached by this project was to use the Go language to replicate the SuperCollider's high-level concept of stream operators as co-routines. Operators are thus represented by goroutines, and all communication between them occurs over channels.

\subsection{Implementation}



\section{Go Language Evaluation}
\label{sec:language}

\section{Go Runtime Evaluation}
\label{sec:runtime}

\end {document}
